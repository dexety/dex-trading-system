% Это основная команда, с которой начинается любой \LaTeX-файл. Она отвечает за тип документа, с которым связаны основные правил оформления текста.
\documentclass{article}

% Здесь идет преамбула документа, тут пишутся команды, которые настраивают LaTeX окружение, подключаете внешние пакеты, определяете свои команды и окружения. В данном случае я это делаю в отдельных файлах, а тут подключаю эти файлы.

% Здесь я подключаю разные стилевые пакеты. Например возможности набирать особые символы или возможность компилировать русский текст. Подробное описание внутри.
\usepackage{packages}

% Здесь я определяю разные окружения, например, теоремы, определения, замечания и так далее. У этих окружений разные стили оформления, кроме того, эти окружения могут быть нумерованными или нет. Все подробно объяснено внутри.
\usepackage{environments}

% Здесь я определяю разные команды, которых нет в LaTeX, но мне нужны, например, команда \tr для обозначения следа матрицы. Или я переопределяю LaTeX команды, которые работают не так, как мне хотелось бы. Типичный пример мнимая и вещественная часть комплексного числа \Im, \Re. В оригинале они выглядят не так, как мы привыкли. Кроме того, \Im еще используется и для обозначения образа линейного отображения. Подробнее описано внутри.
\usepackage{commands}

% Пакет для титульника проекта
\usepackage{titlepage}

% Пакет для кода
\usepackage{listings}

% Здесь задаем параметры титульной страницы
\setUDK{004.942}
% Выбрать одно из двух
% \setToResearch
\setToProgram

\setTitle{Торговая система для крипто-бирж}

% Выбрать одно из трех:
% КТ1 -- \setStageOne
% КТ2 -- \setStageTwo
% Финальная версия -- \setStageFinal
\setStageOne
%\setStageTwo
%\setStageFinal

\setGroup{209}
%сюда можно воткнуть картинку подписи
\setStudentSgn{}

\setStudentDate{17.02.2022}
\setAdvisor{Казаков Евгений Александрович}
\setAdvisorTitle{разработчик}
\setAdvisorAffiliation{Facebook inc.}
\setAdvisorDate{17.02}
\setGrade{10}
%сюда можно воткнуть картинку подписи
\setAdvisorSgn{}
\setYear{2022}


% С этого момента начинается текст документа
\begin{document}

% Эта команда создает титульную страницу
\makeTitlePage

% Здесь будет автоматически генерироваться содержание документа
\tableofcontents

% Данное окружение оформляет аннотацию: краткое описание текста выделенным абзацем после заголовка
\begin{abstract}
Исследование и написание торговых стратегия для децентрализованных бирж на Layer-2\footnote{Технологии масштабирования, позволяющие увеличить скорость и пропускную способность блокчейна. Обычно это делается за счет децентрализованности. Самые распространенные разновидности — rollups (свертки, собирают несколько транзакций в одну и записывают в блокчейн) и sidechain (блокчейн, который опирается на масштабируемую сеть. Часто используют менее децентрализованную и, как следствие, более быстрые консенсусные алгоритмы)}.
\end{abstract}


\section{Введение}

\subsection{Актуальность проблемы}
Сегодня, кого не спроси, все знают, что такое биткоин, или, по крайней мере, говорят, что знают. Разговоры о криптовалютах и их производных в мире не утихают уже несколько лет, но, к сожалению, большинство из них очень поверхностные и не доходят даже до обсуждения принципа работы блокчейна в общих словах.
Несмотря на кажущуюся сложность устройства, 2 самых больших блокчейна (Bitcoin и Ethereum) критически не справляются с нагрузкой, возложенной на них желающими воспользоваться их плюсами. Из-за того, что сеть Эфириума может обрабатывать лишь 15 транзакций в секунду, комиссия, которую нужно заплатить, чтобы транзакция была одной из этих 15 доходит до \$70, что делает любой токен на блокчейне непригодным для использования в качестве обычной фиатной валюты. Чтобы снизить нагрузку на мэйннет\footnote{Основная сеть блокчейна, на которой криптовалюта имеет реальную стоимость. Есть также сети для тестирования разработок. На них валюту можно получить по запросу от специальных адресов}, были разработаны и все еще разрабатываются несколько альтернативных решений, которые одним словом называются Layer-2 решения. Это надстройки над блокчейном, которые увеличивают пропускную способность и скорость в ущерб децентрализованности.
Мы считаем, что пока не придумали более изящного способа достичь тех же результатов, которые дают L2 решения, данная технология будет развиваться, а актуальности нашей темы будет расти.

\subsection{Цели и задачи}
\subsubsection{Цель}
Написать трэйдинг систему, которая сможет стабильно выходить в плюс.
\subsubsection{Задачи}
\begin{itemize}
\item Проведение исследований по стратегиям трэйдинга.
\item Проверка работоспособность стратегий.
\item Создание инфраструктуры, позволяющей взаимодействовать с биржей автоматизированно.
\item Сбор данных и обучение модели.
\item Написание программы, совершающей сделки.
\item Обзор и сравнительный анализ источников и аналогов
\end{itemize}
К сожалению, выбранная нами тема мало освещается в источниках любого вида: никто не захочет делиться прибыльной стратегией. Многое нам приходилось и придется делать с нуля.
\subsection{Статьи}
Тем не менее, существуют статьи, описывающие некоторые возможные подходы к написанию алгоритмов HFT. Например, есть ресурс~\cite{HftBattle}, на котором описывается стратегия маркет-мейкинга, аналог которой мы попытались реализовать. Но материалы такого рода, находящиеся в открытом доступе, с течением времени теряют свою актуальность: если большое количество участников рынка придерживается одной схемы действий, то вскоре она перестает приносить прибыль. По этой причине мы старались не ориентироваться на подобные источники.
\subsection{Документация бирж}
Основным же источником информации для нас служила документация API~\cite{DydxDocs}~\cite{BinanceDocs} бирж, к которым мы подключались. С помощью нее был написан коннектор, инкапсулирующий процесс подключения и взаимодействия с биржей, произведен сбор необходимой информации: список сделок за последний месяц, состояние о счете и т.п.
\subsection{Документация библиотек}
Для машинного обучения мы использовали CatBoost~\cite{CatboostDocs} — это библиотека от Яндекса для градиентного бустинга, надстройки над решающими деревьями. КэтБуст для нас лучшее решение, потому что это самая быстрая библиотека для классификации среди аналогов и проста в использовании.

\subsection{Аналоги}
На крипто валютном рынке существует множество торговых ботов, но информации об их характеристиках и принципах работы практически нет. Мы можем судить об их доходности, лишь по каким-то сомнительным заявлениям или косвенным признакам. В открытом доступе в основном находятся боты, которые предоставляют лишь интерфейс взаимодействия с биржей ~\cite{CryptoTradingBot}~\cite{FreqTrade}: “ручная” покупка и продажа токенов, выставление лимитных ордеров и т.п.. Такие решения не представляют для нас никакого интереса.


\section{Описание функциональных требований к программному проекту}

\subsection{Сбор данных}
Нужно обеспечить удобный механизм сбора исторических данных с бирж, на которых будет тестироваться и обучаться система.

\subsection{Измерение скорости соединения}
Так как счет идет на миллисекунды, мы всегда должны иметь четкое представления, какое время у нас займет отправка и получения пакета данных. Для этого должен быть предусмотрен отдельный модуль, который будет замерять скорость соединения с различными сервисами.

\subsection{Коннектор к бирже}
В программе должны быть коннекторы к биржам. Это класс, в конструктор которого подаются приватные ключи кошельков. После этого можно работать с биржей: смотреть информацию о счете, валютаю, отправлять и отменять ордера. Надо реализовать функционал, который предоставляет апи, чтобы можно было его использовать в трейдинг-стратегиях.

\subsection{Машинное обучение}
Индикаторов, по которым можно строить прогнозы, очень много, поэтому ручными методами не получится подобрать правильное соотношение весов. Здесь нужно машинное обучение, которое принимает на вход пред обработанные данные сделок, а на выход выдает модель, которую можно использовать в трейдинг-стратегии. Важно, чтобы модель была устойчива к тому, что в датасет добавляется или убирается индикаторы.
Примеры использования

\subsection{Коннектор}
Предоставляем класс, который способен взаимодействовать с биржами по API, например:

\begin{itemize}

\item Отправка ордеров
\begin{lstlisting}[language=Python]
connector.send_order(
        symbol=ETH_USD, side=BUY, price=1, quantity=0.1
)
\end{lstlisting}

\item Получение текущих позиций
\begin{lstlisting}[language=Python]
connector.get_our_positions(
        opened=True, symbol=ETH_USD
)
\end{lstlisting}

\item Получение трейдов за определенный промежуток времени
\begin{lstlisting}[language=Python]
connector.get_historical_trades(
        market=BTC_USD,
        begin="2021-12-12 09:00:00",
        end="2021-12-12 12:00:00"
)
\end{lstlisting}

\item Информация о конкретном рынке
\begin{lstlisting}[language=Python]
connector.get_symbol_info(market=ETH_USD)
\end{lstlisting}

\item Измерение скорости
\begin{lstlisting}[language=Python]
speed_measure = SpeedMeasure(connector)
speed_measure.get_connector_funcs_exec_times(
        market=ETH_USD,
        side=BUY,
        iters_num=10,
        filename="connector_funcs_exec_times.json",
    )
\end{lstlisting}

\item Измерение задержки до биржи
\begin{lstlisting}[language=Python]
speed_measure.get_orders_processing_delays(
        market=ETH_USD,
        side=BUY,
        orders_num=10,
        filename="orders_processing_delays.json",
)
\end{lstlisting}

\end{itemize}

\section{Описание нефункциональных требований к программному проекту}

\subsection{Безопасность}
Финансы — чувствительная тема, поэтому наша программа не должна допускать утечек данных о кошельках и приватных ключах. Нужно обеспечить безопасное хранение.

\subsection{Отказоустойчивость}
Во время трейдинга торговая система получает сотни обновлений от разных бирж, их обрабатывает, строит прогнозы и торгует. Нужно сделать так, чтобы система была готова к большим нагрузкам, и поведение было однозначно определено. Еще нужно проработать быстрое отключение торговой системы от торгов, если ее поведение станет неадекватным, и можно было бы быстро закрыть открытые заявки.

\subsection{Скорость}
В трейдинге важна каждая миллисекунда, поэтому цель — минимизировать время обработки, отправки и принятия данных.

\subsection{Переиспользование кода}
Нужно выстроить архитектуру проекта так, чтобы можно было быстро и легко тестировать свои гипотезы, поэтому код, который есть в проекте, должен быть написан так, чтобы его можно было легко понять и переиспользовать в других местах.

\subsection{Масштабируемость}
Система должна быть расширяема на несколько бирж и потенциально работать с большим количеством предсказательных моделей.

\section{Дополнительный результаты}

Помимо задач, которые нам поставил руководитель, мы сделали еще:

\subsection{Визуализация}
Перед тем, как работать с данными, надо понять, как они устроены: попарное распределение классов, плотность каждого из индикаторов. Мы все это сделали. Теперь стало гораздо проще подбирать параметры для модели машинного обучения.

\subsection{Маркет-Мэйкинг стратегия}
Это стратегия, когда мы держим окно из зеркальных заявок на каком-то уровне от индекс-цены. Утверждается, что если нас пробили с одной стороны, то почти сразу пробьют и со второй стороны, и мы окажемся в плюсе. Но все оказалось не так: при пробитии нас с одной стороны, рынок продолжал идти в ту же сторону.

\subsection{Контролирующий бот}
Мы сделали телеграм бота, на которого можно по паролю подписаться и получать обновления состояния аккаунта на бирже. Это полезно, когда ты запускаешь торговую стратегию, куда-то отходишь, но при этом всегда можешь контролировать, что с ней происходит, через телеграм.

\subsection{CI/CD, тесты, линтер}
Любая ошибка в торговой системе может потерять наши деньги, поэтому важно, чтобы весь код всегда был рабочим. Для этого мы все, что смогли, обложили тестами c использованием утилиты PyTest~\cite{Pytest}. Теперь при каждом пулл реквесте у нас запускается тестирующая система, и если какие-то тесты не проходят, то мы запрещаем дальнейший пуш. Реализовали мы это через GitHub Actions~\cite{GitHubActions}.

Еще нам хочется, чтобы весь код был консистеным. Для этого мы используем линтер Black~\cite{Black} и статический анализатор PyLint~\cite{Pylint}. В совокупности эти утилиты поддерживают консистентность нашего кода и могут еще до запуска теста, выдать синтаксические ошибки.

\subsection{Смарт контракты}
С помощью смарт контрактов можно занимать у других людей миллиарды, трейдить на них, и потом возвращать криптовалюту обратно. Звучит заманчиво. Мы написали смарт конракты на Solidity~\cite{Solidity}, и залили их в сеть эфира через Brownie~\cite{Brownie}.



% Здесь автоматически генерируется библиография. Первая команда задает стиль оформления библиографии, а вторая указывает на имя файла с расширением bib, в котором находится информация об источниках.
\bibliographystyle{plainurl}
\bibliography{bibl}

% Здесь текст документа заканчивается
\end{document}
% Начиная с этого момента весь текст LaTeX игнорирует, можете вставлять любую абракадабру.
