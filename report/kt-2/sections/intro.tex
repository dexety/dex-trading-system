\section{Введение}

\subsection{Актуальность проблемы}
Сегодня, кого не спроси, все знают, что такое биткоин, или, по крайней мере, говорят, что знают. Разговоры о криптовалютах и их производных в мире не утихают уже несколько лет, но, к сожалению, большинство из них очень поверхностные и не доходят даже до обсуждения принципа работы блокчейна в общих словах.
Несмотря на кажущуюся сложность устройства, 2 самых больших блокчейна (Bitcoin и Ethereum) критически не справляются с нагрузкой, возложенной на них желающими воспользоваться криптовалютой. Из-за того, что сеть Эфириума может обрабатывать лишь 15 транзакций в секунду, комиссия, которую нужно заплатить, чтобы транзакция была одной из этих 15 доходит до \$70, что делает любой токен на блокчейне непригодным для использования в качестве обменной валюты. Чтобы снизить нагрузку на мэйннет\footnote{Основная сеть блокчейна, на которой криптовалюта имеет реальную стоимость. Есть также сети для тестирования разработок. На них валюту можно получить по запросу от специальных адресов}, были разработаны и все еще разрабатываются несколько альтернативных решений, которые одним словом называются Layer-2 решения. Это надстройки над блокчейном, которые увеличивают пропускную способность и скорость в ущерб децентрализованности.
Мы считаем, что пока не придумали более изящного способа достичь тех же результатов, которые дают L2 решения, данная технология будет развиваться, а актуальности нашей темы будет расти.

\subsection{Цели и задачи}
\subsubsection{Цель}
Написать трэйдинг систему, которая сможет стабильно выходить в плюс.
\subsubsection{Задачи}
\begin{itemize}
\item Проведение исследований по стратегиям трэйдинга.
\item Проверка работоспособность стратегий.
\item Создание инфраструктуры, позволяющей взаимодействовать с биржей автоматизированно.
\item Сбор данных и обучение модели.
\item Написание программы, совершающей сделки.
\item Обзор и сравнительный анализ источников и аналогов
\end{itemize}
К сожалению, выбранная нами тема мало освещается в источниках любого вида: никто не захочет делиться прибыльной стратегией. Многое нам приходилось и придется делать с нуля.
\subsection{Статьи}
Тем не менее, существуют статьи, описывающие некоторые возможные подходы к написанию алгоритмов HFT. Например, есть ресурс~\cite{HftBattle}, на котором описывается стратегия маркет-мейкинга, аналог которой мы попытались реализовать. Но материалы такого рода, находящиеся в открытом доступе, с течением времени теряют свою актуальность: если большое количество участников рынка придерживается одной схемы действий, то вскоре она перестает приносить прибыль. По этой причине мы старались не ориентироваться на подобные источники.
\subsection{Документация бирж}
Основным же источником информации для нас служила документация API~\cite{DydxDocs}~\cite{BinanceDocs} бирж, к которым мы подключались. С помощью нее был написан коннектор, инкапсулирующий процесс подключения и взаимодействия с биржей, произведен сбор необходимой информации: список сделок за последний месяц, состояние о счете и т.п.
\subsection{Документация библиотек}
Для машинного обучения мы использовали CatBoost~\cite{CatboostDocs} — это библиотека от Яндекса для градиентного бустинга, надстройки над решающими деревьями. КэтБуст для нас лучшее решение, потому что это самая быстрая библиотека для классификации среди аналогов и проста в использовании.

\subsection{Аналоги}
На крипто валютном рынке существует множество торговых ботов, но информации об их характеристиках и принципах работы практически нет. Мы можем судить об их доходности, лишь по каким-то сомнительным заявлениям или косвенным признакам. В открытом доступе в основном находятся боты, которые предоставляют лишь интерфейс взаимодействия с биржей ~\cite{CryptoTradingBot}~\cite{FreqTrade}: “ручная” покупка и продажа токенов, выставление лимитных ордеров и т.п.. Такие решения не представляют для нас никакого интереса.