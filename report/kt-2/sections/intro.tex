\section{Введение}

\subsection{Актуальность проблемы}
Все говорят о криптовалютах, но толком никто ничего о них не знает. Мы решили разобраться, как устроен блокчейн и технологии вокруг него, какие есть проблемы и как с ними справляются.

Одна из проблем больших блокчейнов, таких как Bitcoin и Ethereum, -- высокая загруженность. Они критически не справляются с нагрузкой. Из-за того, что сеть Эфириума может обрабатывать лишь 15 транзакций в секунду, комиссия, которую нужно заплатить, чтобы транзакция была одной из этих 15, доходит до \$100, что делает любой токен на блокчейне непригодным для использования в качестве обменной валюты.


Чтобы снизить нагрузку на мэйннет\footnote{Основная сеть блокчейна, на которой криптовалюта имеет реальную стоимость. Есть также сети для тестирования разработок. На них валюту можно получить по запросу от специальных адресов}, были разработаны и все еще разрабатываются несколько альтернативных решений, которые одним словом называются \texttt{Layer-2} решения. Это надстройки над блокчейном, которые увеличивают пропускную способность и скорость в ущерб децентрализованности.


Мы считаем, что пока не придумали более изящного способа достичь тех же результатов, которые дают L2 решения, данная технология будет развиваться, а актуальности нашей темы будет расти.


\subsection{Цели и задачи}
\subsubsection{Цель}
Написать трейдинг систему, которая сможет стабильно выходить в плюс.
\subsubsection{Задачи}
\begin{itemize}
\item Проведение исследований по стратегиям трэйдинга.
\item Проверка работоспособность стратегий.
\item Создание инфраструктуры, позволяющей взаимодействовать с биржей автоматизированно.
\item Сбор данных и обучение модели.
\item Написание программы, совершающей сделки.
\item Обзор и сравнительный анализ источников и аналогов
\end{itemize}
К сожалению, выбранная нами тема мало освещается в источниках любого вида: никто не захочет делиться прибыльной стратегией. Многое нам приходилось и придется делать с нуля.
\subsection{Статьи}
Тем не менее, существуют статьи, описывающие некоторые возможные подходы к написанию алгоритмов HFT. Например, есть ресурс~\cite{HftBattle}, на котором описывается стратегия маркет-мейкинга, аналог которой мы попытались реализовать. Но материалы такого рода, находящиеся в открытом доступе, с течением времени теряют свою актуальность: если большое количество участников рынка придерживается одной схемы действий, то вскоре она перестает приносить прибыль. По этой причине мы старались не ориентироваться на подобные источники.
\subsection{Документация бирж}
Основным же источником информации для нас служила документация API~\cite{DydxDocs}~\cite{BinanceDocs} бирж, к которым мы подключались. С помощью нее был написан коннектор, инкапсулирующий процесс подключения и взаимодействия с биржей, произведен сбор необходимой информации: список сделок за последний месяц, состояние о счете и т.п.
\subsection{Документация библиотек}
Для машинного обучения мы использовали CatBoost~\cite{CatboostDocs} — это библиотека от Яндекса для градиентного бустинга, надстройки над решающими деревьями. КэтБуст для нас лучшее решение, потому что это самая быстрая библиотека для классификации среди аналогов и проста в использовании.

\subsection{Аналоги}
На крипто валютном рынке существует множество торговых ботов, но информации об их характеристиках и принципах работы практически нет. Мы можем судить об их доходности, лишь по каким-то сомнительным заявлениям или косвенным признакам. В открытом доступе в основном находятся боты, которые предоставляют лишь интерфейс взаимодействия с биржей ~\cite{CryptoTradingBot}~\cite{FreqTrade}: “ручная” покупка и продажа токенов, выставление лимитных ордеров и т.п.. Такие решения не представляют для нас никакого интереса.

Так достойных аналогов в свободном доступе нет, функциональные и нефункциональные требования нам пришлось придумывать самим. К счастью, они достаточно интуитивны.

\subsection{Описание функциональных требований к программному проекту}

\subsubsection{Коннектор к бирже}

В программе должны быть коннекторы к биржам. Это класс, в конструктор которого подаются приватные ключи кошельков. После этого можно работать с биржей: смотреть информацию о счете, валютаю, отправлять и отменять ордера. Надо реализовать функционал, который предоставляет апи, чтобы можно было его использовать в трейдинг-стратегиях.

Примеры использования функций класса, который способен взаимодействовать с биржами по API:

\begin{itemize}

\item Отправка ордеров
\begin{verbatim}
connector.send_order(
        symbol=ETH_USD, side=BUY, price=1, quantity=0.1
)
\end{verbatim}

\item Получение текущих позиций
\begin{verbatim}
connector.get_our_positions(
        opened=True, symbol=ETH_USD
)
\end{verbatim}

\item Информация о конкретном рынке
\begin{verbatim}
connector.get_symbol_info(market=ETH_USD)
\end{verbatim}

\end{itemize}

\subsubsection{Сбор данных}
Нужно обеспечить удобный механизм сбора исторических данных с бирж, на которых будет тестироваться и обучаться система. Удобно реализовать такие функции в коннекторе, ведь у него уже есть соединение с биржей.

Пример функции для сбора данных:
\begin{itemize}
\item Получение трейдов за определенный промежуток времени
\begin{verbatim}
connector.get_historical_trades(
        market=BTC_USD,
        begin="2021-12-12 09:00:00",
        end="2021-12-12 12:00:00"
)
\end{verbatim}
\end{itemize}

Также в коннекторе может быть функционал, позволяющий собирать данные в прямом эфире и сохранять их в архив, если эти исторические данные не предоставляет биржа.

\subsubsection{Измерение скорости соединения}
Так как счет идет на миллисекунды, мы всегда должны иметь четкое представления, какое время у нас займет отправка и получения пакета данных. Для этого должен быть предусмотрен отдельный модуль, который будет замерять скорость соединения с различными сервисами. Он тоже основывается на коннекторе, так как он уже умеет соединяться с биржей.

Примеры использования функций класса для измерения скорости коннектора:
\begin{itemize}
\item Измерение скорости
\begin{verbatim}
speed_measure = SpeedMeasure(connector)
speed_measure.get_connector_funcs_exec_times(
        market=ETH_USD,
        side=BUY,
        iters_num=10,
        filename="connector_funcs_exec_times.json",
    )
\end{verbatim}

\item Измерение задержки до биржи
\begin{verbatim}
speed_measure.get_orders_processing_delays(
        market=ETH_USD,
        side=BUY,
        orders_num=10,
        filename="orders_processing_delays.json",
)
\end{verbatim}
\end{itemize}

\subsubsection{Машинное обучение}
Индикаторов, по которым можно строить прогнозы, очень много, поэтому ручными методами не получится подобрать правильное соотношение весов. Здесь нужно машинное обучение, которое принимает на вход пред обработанные данные сделок, а на выход выдает модель, которую можно использовать в трейдинг-стратегии. Важно, чтобы модель была устойчива к тому, что в датасет добавляется или убирается индикаторы.
Примеры использования

\subsection{Описание нефункциональных требований к программному проекту}

\subsubsection{Безопасность}
Финансы — чувствительная тема, поэтому наша программа не должна допускать утечек данных о кошельках и приватных ключах. Нужно обеспечить безопасное хранение.

\subsubsection{Отказоустойчивость}
Во время трейдинга торговая система получает сотни обновлений от разных бирж, их обрабатывает, строит прогнозы и торгует. Нужно сделать так, чтобы система была готова к большим нагрузкам, и поведение было однозначно определено. Еще нужно проработать быстрое отключение торговой системы от торгов, если ее поведение станет неадекватным, и можно было бы быстро закрыть открытые заявки.

\subsubsection{Скорость}
В трейдинге важна каждая миллисекунда, поэтому цель — минимизировать время обработки, отправки и принятия данных.

\subsubsection{Переиспользование кода}
Нужно выстроить архитектуру проекта так, чтобы можно было быстро и легко тестировать свои гипотезы, поэтому код, который есть в проекте, должен быть написан так, чтобы его можно было легко понять и переиспользовать в других местах.

\subsubsection{Масштабируемость}
Система должна быть расширяема на несколько бирж и потенциально работать с большим количеством предсказательных моделей.