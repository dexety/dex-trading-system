\section{Market Making}
\subsection{Что это такое}

Когда человек приходит на биржу, он хочет купить актив по рыночной цене.
Если какая-то биржа будет продавать биткоины по 11k\$, а все остальные по 10k\$, то на вряд ли у нее кто-то купит биткоины.


Для решения такого дисбаланса существуют маркет-мейкеры. Они поддерживают баланс спроса и предложения так, чтобы цена покупки и продажа была близка к рыночной. Чем больше биржа, тем больше нужно денег, чтобы быть маркет-мейкером, поэтому они спонсируются биржей и часто имеют привилегированные условия торговли. Мы не оказались в ряду этих счастливчиков, поэтому торговали на равных условиях.

\subsection{Как на этом можно заработать}
Заработать можно в предположении краткосрочно высоко-инерционного рынка. Как это работает:

\definition \textbf{Пробитием} будем называть ту заявку, которую мы не успели отменить, и по ней у нас открылась позиция.

\begin{algorithm}
Market-Making orders
\begin{enumerate}

    \item \label{mm:init} Выставляем лимитные заявки в обе стороны на $\pm \Delta_1$ от индекс-цены.
    
    \item \label{mm:swap} Когда индекс-цена изменяется, переставляем заявки на ту же самую $\pm \Delta_1$, но уже от новой цены.
    
    \item Если переставиться успели, и нас не пробили, то переходим к шагу \ref{mm:swap}
    
    \item Неумаляя общности нас пробили на покупку по цене $p$. Отменяем все заявки
    
    \item Выставляем заявку на продажу на $\Delta_2$ от цены, по которой нас пробили на покупку
    
    \item \label{mm:win} Когда заявка исполнилась, возвращаемся к шагу \ref{mm:init}
    
\end{enumerate}
\end{algorithm}

В итоге на шаге \ref{mm:win} мы заработает $\Delta_2 \cdot p$.

\subsection{Реализация}
\hyperlink{https://github.com/dexety/dex-trading-system/tree/main/research/lp-0003-market-making}{Стратегия в репозитории}

Программа работает в трех потоках, которые нужны для: 

\begin{enumerate}

\item Выставление ордеров
\begin{enumerate}
    \item В этом потоке мы получаем новый трейды.
    \item Если цена нового трейда не такая, как у прошлого, и цена ордера, который надо выставить не такая, как у нас сейчас стоит, то мы отменяем текущие ордера и выставляем новые.
    \item Перестановка ордеров переходит не через сначала отмену, старых, а потом выставления новых ордеров, а непосредственно через переставление позиций аргументом \texttt{cancel\_id} в функции \texttt{create\_order}. Это позволило сократить количество запросов в два раза, а значит ошибка \texttt{Too many requests} будет встречаться реже.
\end{enumerate}

\item Получение обновлений ордербука
\begin{enumerate}
    \item В предыдущем потоке во время выставления и отмены ордеров новые обновления по вебсокету не приходят, но обновления ордер бука нам нужны, потому что по нему мы определяем цену оредров.
    \item Поэтому получения ордер бука вынесено в отдельный поток, чтобы всегда именть свежий стакан.
\end{enumerate}

\item Проверки положение ордеров
\begin{enumerate}
    \item Бывает так, что новые трейды не приходят, но окно спреда ордер бука меняется.
    \item Нам важно наше окно трейдов держать строго на $\pm$ какой-то спред вокруг бидсов и асков.
    \item Поэтому мы раз в какое-то время проверяем этот баланс, и если происходит дисбаланс, то мы обновляем наши ордера.
\end{enumerate}

\end{enumerate}

\subsection{Результаты}
Мы торговали на бирже \texttt{DyDx}. Она входит в топ самых популярных децентрализованных бирж. Количество сделок в месяц в ней около 30 тысяч и по биткоинку, и по эфиру. На самом деле это очень мало: в среднем $0.3$ сделки в секунду, то есть одна сделка в 3-4 секунды. Соответственно рынок совсем не высоко-инерционный, поэтому после того, как нас пробили в одну сторону, до пробития во вторую, может пройти несколько десятков минут. Это превышает наш таймаут, потому что если мы слишком долго будем держать открытую позицию, то повышается вероятность, что рынок пойдет в обратную сторону и мы потеряем еще больше денег.

Еще часто бывали случаи, когда рынок отскакивает в нужную нам сторону, но на величину меньше комиссии, соответственно, мы теряем деньги.

\subsection{Выводы}

Если бы у нас была нулевая или близка к нулевой комиссия, то мы бы были в плюсе. Но с той, что у нас есть, в минусе.