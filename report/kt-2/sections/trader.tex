\section{Трейдер}

\href{https://github.com/dexety/dex-trading-system/blob/main/strategy/arbitrage/trader.py}{Трейдер в репозитории}

\begin{definition}
\textbf{Трейдер} -- программа, которая на основе готовой стратегии, совершает сделки.
\end{definition}

\subsection{Использованные технологии}

\subsubsection{Асинхронная функция}

\begin{definition}
\textbf{Асинхронная функция} -- функция, которая не блокирует основой поток программы.
\end{definition}

Наш проект написан на питоне, а он одно-поточный. Поэтому нужно было пользоваться всеми возможностями асинхронности. Для этого в питоне есть очень простой интерфейс:

\begin{verbatim}
async def slow_calculation():
    return ...
    
async def hard_calculation():
    return ...
    
def main():
    loop = asyncio.get_event_loop()
    loop.create_task(
        slow_calculation(), name="slow_calculation"
    )
    loop.create_task(
        hard_calculation(), name="hard_calculation"
    )
    loop.run_forever()
\end{verbatim}

Здесь мы создали две задачи: \texttt{slow\_calculation} и  \texttt{hard\_calculation}, которые будут выполняться асинхронно, то есть не мешать друг другу своим вычислениями.

\subsubsection{asyncio.Event}

В трейдере, помимо асинхронных функций, используется класс \texttt{asyncio.Event}, позволяющий ждать задачам, пока событие не произойдет в другой задаче и элемент класса не будет разблокирован.

\begin{verbatim}
async def waiter(event):
    print('waiting for it ...')
    await event.wait()
    print('... got it!')

async def main():
    # Create an Event object.
    event = asyncio.Event()

    # Spawn a Task to wait until 'event' is set.
    waiter_task = asyncio.create_task(waiter(event))

    # Sleep for 1 second and set the event.
    await asyncio.sleep(1)
    event.set()

    # Wait until the waiter task is finished.
    await waiter_task

asyncio.run(main())
\end{verbatim}

В данном примере функция \texttt{waiter} не завершится до тех пор, пока функция \texttt{main} не разблокирует событие строчкой \texttt{event.set()}, потому что \texttt{waiter} находится в ожидании после строчки \texttt{await event.wait()}.

С помощью этого класса можно синхронизировать работу алгоритма торговли с ответами от биржи.

\subsection{Устройство трейдера}

Используемые параметры:
\begin{itemize}
    \item \href{https://github.com/dexety/dex-trading-system/blob/ca0370d602f2dfa05262b9b8574002f965ac1502/research/ib-0002-cross-analysis/trader.py#L52}{\texttt{trailing\_percent}} -- процент, на который цена stop-loss ордера отстает от рыночной при отправке trailing-stop ордера.
    \item \href{https://github.com/dexety/dex-trading-system/blob/ca0370d602f2dfa05262b9b8574002f965ac1502/research/ib-0002-cross-analysis/trader.py#L54}{\texttt{profit\_threshold}} -- процент, на который цена лимитки, выставленной по take-profit ордеру отличается от рыночной.
    \item \href{https://github.com/dexety/dex-trading-system/blob/ca0370d602f2dfa05262b9b8574002f965ac1502/research/ib-0002-cross-analysis/trader.py#L55}{\texttt{sec\_to\_wait}} -- количество секунд, через которое цикл торговли считается завершенным, если ни take-profit, ни trailing-stop ордеры не были исполнены.
\end{itemize}

Трейдер состоит из двух асинхронных задач и вспомогательных функций. 

Из ключевых вспомогательных функций нужно отметить функцию \href{https://github.com/dexety/dex-trading-system/blob/main/research/ib-0002-cross-analysis/trader.py#L303}{\texttt{reset}}, которая переводит все поля класса в состояние для ожидания скачка на бинансе. Она вызывается после успешного завершения цикла торговли, или после отмены ордеров и закрытия позиции в случае, когда какой либо ордер не смог исполниться.

Основные асинхронные задачи:
\begin{itemize}
    \item \texttt{listen\_binance}
    \item \texttt{account\_listener}
\end{itemize}

Пройдемся по каждой

\subsubsection{listen\_binance}

\href{https://github.com/dexety/dex-trading-system/blob/main/research/ib-0002-cross-analysis/trader.py#L167}{Функция в репозитории}

\begin{definition}
\textbf{Слушать [что-то]} -- значит подключиться по веб-сокету к какому-то серверу и получать от него обновления.
\end{definition} 

Эта функция реализует логику торговли. В этом трейдере по техническим причинам не получилось использовать функционал коннектора для бинанса, поэтому подписываться на обновления пришлось в ручную. Функция держит окно трейдов на одном направлении фиксированной длины, и если перепад цены превысил порог, то запускается алгоритм торговли:

\begin{enumerate}
    \item Отправляется маркет ордер. Функция входит в режим ожидания с помощью \texttt{asyncio.Event} и ждет, пока с биржи придет ответ об исполнении ордера. Если с биржи приходит сообщение об отмене ордера, то вызывается функция \texttt{reset} и трэйдер возвращается в состояние ожидания скачка цены на бинансе.
    \item Когда \texttt{account\_listener} сообщил об успешном выполнении маркет ордера и сохранил цену по которой он исполнился, трейдер выставляет \texttt{take-profit} и \texttt{trailing-stop} ордеры с заданными при инициализации класса параметрами и на протяжении \texttt{sec\_to\_wait} секунд ожидает исполнения хотя бы одного из них.
    \begin{itemize}
        \item Если за \texttt{sec\_to\_wait} секунд ордеры не закрылись, то они отменяются. Далее закрывается позиция в которую мы зашли маркетом, и когда с биржи приходит сообщение о закрытии позиции, цикл торговли считается завершенным.
        \item Если какой-то из ордеров исполнился, то мы в любом случае отменяем оставшийся. Если исполнился лимит, то мы в плюсе, если трэйлинг стоп, то, скорее всего, в минусе. После отмены цикл считается завершенным.
    \end{itemize}
    
\end{enumerate}


\subsubsection{account\_listener}

\href{https://github.com/dexety/dex-trading-system/blob/main/research/ib-0002-cross-analysis/trader.py#L365}{Функция в репозитории}

Эта задача слушает изменения в наших ордерах и позициях, обновляет цену, по которой исполнился маркет, выставляет флаги и разблокирует нужные события, на которых спит функция \texttt{listen\_binance}, тем самым обеспечивая её синзронизацию с биржей.
