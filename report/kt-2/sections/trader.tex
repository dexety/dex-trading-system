\section{Трейдер}
\hyperlink{https://github.com/dexety/dex-trading-system/blob/main/strategy/arbitrage/trader.py}{Трейдер в репозитории}

\definition \textbf{Трейдер} -- программа, которая на основе готовой стратегии, совершает сделки.

\subsection{Асинхронность}

\definition \textbf{Асинхронная функция} -- функция, которая не блокирует основой поток программы.

Наш проект написан на питоне, а он одно-поточный. Поэтому нужно было пользоваться всеми возможностями асинхронности. Для этого в питоне есть очень простой интерфейс:

\begin{verbatim}
async def slow_calculation():
    return ...
    
async def hard_calculation():
    return ...
    
def main():
    loop = asyncio.get_event_loop()
    loop.create_task(
        slow_calculation(), name="slow_calculation"
    )
    loop.create_task(
        hard_calculation(), name="hard_calculation"
    )
    loop.run_forever()
\end{verbatim}

Здесь мы создали две задачи: \texttt{slow\_calculation} и  \texttt{hard\_calculation}, которые будут выполняться асинхронно, то есть не мешать друг другу своим вычислениями.


\subsection{Устройство трейдера}

Трейдер состоит из трех асинхронных задач

\begin{enumerate}
    \item \texttt{listen\_binance}
    \item \texttt{listen\_our\_trades}
    \item \texttt{close\_positions}
\end{enumerate}

Пройдемся по каждой

\subsubsection{\texttt{listen\_binance}}
\definition \textbf{Слушать [что-то]} -- значит подключиться по веб-сокету к какому-то серверу и получать от него обновления.

Эта задача создает тред-листенер в Бинанс-коннекторе с хендлером \texttt{binance\_trade\_listener}, который реализует арбитражную стратегию. Когда мы получаем сигнал на покупку или продажу, то выставляем заявку и асинхронно укладываем спать эту задачу на таймаут \texttt{sec\_to\_wait}.


\subsubsection{\texttt{listen\_our\_trades}}

Эта задача добавляет подписку на состояние нашего счета на \texttt{Dydx} с хендлером \texttt{accout\_listener}, который в случае открытых позиций на нашем аккаунте, добавляет страхующую заявку в обратном направлении на \texttt{trailing\_percent} процентов от индекс-цены, которая переставляется каждый раз, когда меняется индекс-цена. Эта заявка нужна для того, чтобы застраховаться от резких отскоков в обратную сторону, на которых мы потеряем деньги.

Когда лимитная заявка, которая была выставлена стратежным хендлером, сыграла, то листенер аккаунта закрывает все активные заявки.

\subsubsection{\texttt{close\_positions}}

Эта задача закрывает все открытые позиции и активные заявки, если истек заданный таймаут.