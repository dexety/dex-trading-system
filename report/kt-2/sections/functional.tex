\section{Описание функциональных требований к программному проекту}

\subsection{Сбор данных}
Нужно обеспечить удобный механизм сбора исторических данных с бирж, на которых будет тестироваться и обучаться система.

\subsection{Измерение скорости соединения}
Так как счет идет на миллисекунды, мы всегда должны иметь четкое представления, какое время у нас займет отправка и получения пакета данных. Для этого должен быть предусмотрен отдельный модуль, который будет замерять скорость соединения с различными сервисами.

\subsection{Коннектор к бирже}
В программе должны быть коннекторы к биржам. Это класс, в конструктор которого подаются приватные ключи кошельков. После этого можно работать с биржей: смотреть информацию о счете, валютаю, отправлять и отменять ордера. Надо реализовать функционал, который предоставляет апи, чтобы можно было его использовать в трейдинг-стратегиях.

\subsection{Машинное обучение}
Индикаторов, по которым можно строить прогнозы, очень много, поэтому ручными методами не получится подобрать правильное соотношение весов. Здесь нужно машинное обучение, которое принимает на вход пред обработанные данные сделок, а на выход выдает модель, которую можно использовать в трейдинг-стратегии. Важно, чтобы модель была устойчива к тому, что в датасет добавляется или убирается индикаторы.
Примеры использования

\subsection{Коннектор}
Предоставляем класс, который способен взаимодействовать с биржами по API, например:

\begin{itemize}

\item Отправка ордеров
\begin{lstlisting}[language=Python]
connector.send_order(
        symbol=ETH_USD, side=BUY, price=1, quantity=0.1
)
\end{lstlisting}

\item Получение текущих позиций
\begin{lstlisting}[language=Python]
connector.get_our_positions(
        opened=True, symbol=ETH_USD
)
\end{lstlisting}

\item Получение трейдов за определенный промежуток времени
\begin{lstlisting}[language=Python]
connector.get_historical_trades(
        market=BTC_USD,
        begin="2021-12-12 09:00:00",
        end="2021-12-12 12:00:00"
)
\end{lstlisting}

\item Информация о конкретном рынке
\begin{lstlisting}[language=Python]
connector.get_symbol_info(market=ETH_USD)
\end{lstlisting}

\item Измерение скорости
\begin{lstlisting}[language=Python]
speed_measure = SpeedMeasure(connector)
speed_measure.get_connector_funcs_exec_times(
        market=ETH_USD,
        side=BUY,
        iters_num=10,
        filename="connector_funcs_exec_times.json",
    )
\end{lstlisting}

\item Измерение задержки до биржи
\begin{lstlisting}[language=Python]
speed_measure.get_orders_processing_delays(
        market=ETH_USD,
        side=BUY,
        orders_num=10,
        filename="orders_processing_delays.json",
)
\end{lstlisting}

\end{itemize}