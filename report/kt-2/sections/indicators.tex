\section{Индикаторы}

После маркет-мейкинга мы решили построить модель машинного обучения для предсказания направления рынка. Для построения модели мы использовали индикаторы, которые зависят от трэйдов в окне. Класс \href{https://github.com/dexety/dex-trading-system/blob/ca0370d602f2dfa05262b9b8574002f965ac1502/utils/indicators.py#L5}{\texttt{Indicators}} содержит функции, заполняющие значения фичей и столбца таргета: \texttt{fill\_features\_values} и \texttt{fill\_target\_values}.

\subsection{\texttt{fill\_target\_values}}
\href{https://github.com/dexety/dex-trading-system/blob/ca0370d602f2dfa05262b9b8574002f965ac1502/utils/indicators.py#L15}{Функция в репозитории}

Функция принимает словарь для заполнения значений, два окна трэйдов и параметры ордеров \texttt{stop\_profit} и \texttt{stop\_loss}. Функция считает, исполнились бы эти ордеры и выставляет соответствующее значение в словарь для заполнения.

\subsection{\texttt{fill\_features\_values}}
\href{https://github.com/dexety/dex-trading-system/blob/ca0370d602f2dfa05262b9b8574002f965ac1502/utils/indicators.py#L48}{Функция в репозитории}

\begin{designation}
Окно длины $t$ -- окно, в котором есть все трэйды, случившиеся не больше чем за $t$ секунд до какого-то момента. Обозначаем как \textit{window}.
\end{designation}

\begin{designation}
    \texttt{window[i]} -- i-ый элементы в окне.
\end{designation}
\begin{designation}
    \texttt{window[i].price} и \texttt{window[i].volume} -- цена и объем i-й сделки в окне соответсвенно.
\end{designation}

Функция принимает словарь для заполнения, окно трэйдов, и два списка вариантов числовых параметров фичей $n$ и $t$. За $n$ всегда будем обозначать количество сделок в окне. Функция заполняет словарь значениями фичей, которые представлены ниже:

\begin{enumerate}
    \item \texttt{seconds\_since\_midnight} -- количество секунд с начала дня.
    \item \texttt{seconds\_since\_n\_trades\_ago} -- количество секунд, прошедших с первого трэйда в окне из $n$ последних трэйдов.
    
    \textbf{Псевдо-формула:} \texttt{(window.end\_time - window[-n].time).to\_seconds()}

    \item \texttt{WI\_exp\_moving\_average} -- экспоненциальное среднее в окне длины $t$. 
    
    \textbf{Псевдо-формула:} \texttt{$\alpha$ (window[-1].price + $(1 - \alpha)$ window[-2].price + \dots + $(1 - \alpha)^{n - 1}$ window[-n].price)}

    \item \texttt{WI\_weighted\_moving\_average} -- взвешанное среднее в окне длины $t$.

    \textbf{Псевдо-формула:} $\frac{\sum\limits_{i=0}^{n} \texttt{window[i].price} \; \cdot \; \texttt{window[i].volume}}{\sum\limits_{i=0}^{n}\texttt{window[i].size}}$

    \item \texttt{WI\_trade\_amount} -- количество сделок в окне длины $t$.

    \textbf{Псевдо-формула:} n

    \item \texttt{WI\_trade\_volume} -- суммарный объем сделок в окне длины $t$.

    \textbf{Псевдо-формула:} $\sum\limits_{i=0}^{n}\texttt{window[i].volume}$

    \item \texttt{WI\_open\_close\_diff} -- \textbf{частное} между ценой закрытия и ценой открытия в окне длины $t$.

    \textbf{Псевдо-формула:} \texttt{window[-1].price / window[0].price}

    \item \texttt{WI\_stochastic\_oscillator} -- стохастический осцилятор.

    \textbf{Псевдо-формула:} $\frac{\texttt{CUR} - \texttt{MIN}}{\texttt{MAX} - \texttt{MIN}}$, где \texttt{CUR} - цена последней сделки, а \texttt{MIN} и \texttt{MAX} - наименьшая и наибольшая цена сделки соответственно в окне длины $t$.

\end{enumerate}

Вторая фича зависит от значения $n$ и от направления трэйдов в окне, поэтому для каждой комбинации направления и параметра $n$ в таблице будет свой столбец.

Последние 5 фичей зависят от $t$ и от направления трэйдов в окне, поэтому для каждой комбинации направления, параметра $t$ и фичи в таблице будет свой столбец.

В послдедствии мы решили еще дополнительно делить все скользящие средние на среднее арифметическое цен всех сделок в окне. 

\textbf{Псевдо-формула:} $\texttt{WI\_moving\_average} \; \cdot \; \frac{n}{\sum\limits_{i=0}^{n - 1} window[i].price}$ 

Это сделано для того, чтобы при глобальном изменении стоимости валюты модель продолжала работать.