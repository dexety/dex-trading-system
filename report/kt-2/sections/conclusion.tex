\section{Заключение}
\subsection{Результаты}
\subsubsection{Коннекторы}
Для взаимодействия с биржей был написан класс, использующий API. Класс предоставляет множество различных методов для алгоритмического трейдинга и может быть использован во многих проектах похожей направленности.
\subsubsection{Market Making}
Попробовали реализовать стратегию market making'a, которая не очень хорошо сработала в наших реалиях, но может быть успешной в рамках децентрализованной торговли, если снизятся комиссии в ETH-2 (очень этого ждем).
\subsubsection{Торговля на основе индикаторов}
Была написана стратегия торговли, базирующаяся на информации об индикаторах. На основе информации о них с помощью библиотеки катбуст мы попытались предсказывать направление движения рынка. Наша стратегия не очень сработала, так как классы были трудно различимы.
\subsubsection{Арбитраж}
Предприняли попытку реализовать стратегию, которая торгует исходя из курса на другой бирже. Для этого написали инструмент, позволяющий рассчитать теоретическую прибыль. Попробовали отделять теоретически прибыльные сигналы от убыточных. 

Написали трейдера для этой стратегии, он слушает вебсокет Binance и торгует на dydx. Вот что-что, а трейдер получился не плохим, очень идиоматично написан, использует немного ресурсов и прост в понимании. Вот бы еще и стратегия работала.

\subsection{Дальнейшие перспективы}
\subsubsection{Разработка инфраструктуры для сбора информации}
По ходу написания и тестирования различных стратегий нами было написано множество различных инструментов для сбора и анализа, которые можно развить в полноценную библиотеку для сбора и анализа данных информации с основных бирж.
\subsubsection{Улучшение имеющихся стратегий}
Можно попробовать доделать уже имеющиеся стратегии. Существует очень много вариантов апгрейда наших наработок. Рассмотрим каждую из стратегий.
\begin{itemize}
    \item \textbf{Market Making}
    \item \textbf{Индикаторы}
    
    Можно придумать новые индикаторы, которые будут использоваться при классификации. Можно обратиться к уже имеющимся вариантам и подыскать там что-то подходящее. Рассмотрение других индикаторов, возможно, приведет к лучшему отделению классов друг от друга. 
    
    Также не стоит забывать о большом количестве моделей для классификации, которые можно использовать для поставленных задач. Быть может, какая-то из них даст нам лучший результат.
    
    В нашей реализации мы никак не использовали данные со стакана. Дальнейшее развитие может заключаться именно в этом: как-то анализировать не только совершенные сделки, но и смотреть, как ведут себя все участники рынка.
    \item \textbf{Арбитраж}
    
    В этой стратегии стоит работать в направлении поиска новых фичей, которые могли бы помочь отделить теоретические успешные сигналы от провальных. Также можно посмотреть в сторону разработки инструмента для подбора ключевых параметров стратегии. Можно слушать несколько бирж и пробовать каким-то образом усреднять сигналы. 
    
    Следует оптимизировать трейдер, возможно, переписать его на другой, более быстрый язык программирования, ведь здесь нам важна каждая миллисекунда.
    
    Попытаться найти географическое положение, в котором задержка до сигнальной биржи и до той, на которой мы торгуем, будет минимальная. В этой стратегии такой характер задержки носит принципиальный характер.

\subsubsection{Поиск новых бирж и инструментов}
Конечно, всегда можно попытаться найти какую-то свежую биржу, на которой еще не расплодились боты всяких HFT фондов, и попробовать уже написанные стратегии. Вполне вероятно, что какое-то время они могут приносить деньги. Главное вовремя понять, когда это закончиться. И следует обратить внимание на другие инструменты и валюты, которые мы обошли стороной. На них успех тоже возможен.

\subsubsection{Переход к по-настоящему децентрализованным технологиям}
Можно направить проект немного в другое русло и рассмотреть торговлю непосредственно в сети эфира. Это уже влечет написание смарт-контрактов и тому подобное. Надо основательно разобраться в блокчейне и изучить технологии. Но тогда нам откроется множество новых подходов к алгоритмическому трейдингу. Они могут быть крайне прибыльными.
    
\end{itemize}
