\section{Дополнительные результаты}

Помимо задач, которые нам поставил руководитель, мы сделали еще:

\subsection{Визуализация}
Перед тем, как работать с данными, надо понять, как они устроены: попарное распределение классов, плотность каждого из индикаторов. Мы все это сделали. Теперь стало гораздо проще подбирать параметры для модели машинного обучения.

\subsection{Контролирующий бот}
Мы сделали телеграм бота, на которого можно по паролю подписаться и получать обновления состояния аккаунта на бирже. Это полезно, когда ты запускаешь торговую стратегию, куда-то отходишь, но при этом всегда можешь контролировать, что с ней происходит, через телеграм.

\subsection{CI/CD, тесты, линтер}
Любая ошибка в торговой системе может потерять наши деньги, поэтому важно, чтобы весь код всегда был рабочим. Для этого мы все, что смогли, обложили тестами c использованием утилиты PyTest~\cite{Pytest}. Теперь при каждом пулл реквесте у нас запускается тестирующая система, и если какие-то тесты не проходят, то мы запрещаем дальнейший пуш. Реализовали мы это через GitHub Actions~\cite{GitHubActions}.

Еще нам хочется, чтобы весь код был консистеным. Для этого мы используем линтер Black~\cite{Black} и статический анализатор PyLint~\cite{Pylint}. В совокупности эти утилиты поддерживают консистентность нашего кода и могут еще до запуска теста, выдать синтаксические ошибки.

\subsection{Смарт контракты}
С помощью смарт контрактов можно занимать у других людей миллиарды, трейдить на них, и потом возвращать криптовалюту обратно. Звучит заманчиво. Мы написали смарт конракты на Solidity~\cite{Solidity}, и залили их в сеть эфира через Brownie~\cite{Brownie}.