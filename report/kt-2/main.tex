% Это основная команда, с которой начинается любой \LaTeX-файл. Она отвечает за тип документа, с которым связаны основные правил оформления текста.
\documentclass{article}

% Здесь идет преамбула документа, тут пишутся команды, которые настраивают LaTeX окружение, подключаете внешние пакеты, определяете свои команды и окружения. В данном случае я это делаю в отдельных файлах, а тут подключаю эти файлы.

% Здесь я подключаю разные стилевые пакеты. Например возможности набирать особые символы или возможность компилировать русский текст. Подробное описание внутри.
\usepackage{packages}

% Здесь я определяю разные окружения, например, теоремы, определения, замечания и так далее. У этих окружений разные стили оформления, кроме того, эти окружения могут быть нумерованными или нет. Все подробно объяснено внутри.
\usepackage{environments}

% Здесь я определяю разные команды, которых нет в LaTeX, но мне нужны, например, команда \tr для обозначения следа матрицы. Или я переопределяю LaTeX команды, которые работают не так, как мне хотелось бы. Типичный пример мнимая и вещественная часть комплексного числа \Im, \Re. В оригинале они выглядят не так, как мы привыкли. Кроме того, \Im еще используется и для обозначения образа линейного отображения. Подробнее описано внутри.
\usepackage{commands}

% Пакет для титульника проекта
\usepackage{titlepage}

% Пакет для кода
\usepackage{listings}

% Здесь задаем параметры титульной страницы
\setUDK{004.942}
% Выбрать одно из двух
% \setToResearch
\setToProgram

\setTitle{Торговая система для крипто-бирж}

% Выбрать одно из трех:
% КТ1 -- \setStageOne
% КТ2 -- \setStageTwo
% Финальная версия -- \setStageFinal
% \setStageOne
\setStageTwo
%\setStageFinal

\setGroup{209}
%сюда можно воткнуть картинку подписи
\setStudentSgn{}

\setStudentDate{17.02.2022}
\setAdvisor{Казаков Евгений Александрович}
\setAdvisorTitle{разработчик}
\setAdvisorAffiliation{Facebook inc.}
\setAdvisorDate{17.02}
\setGrade{10}
%сюда можно воткнуть картинку подписи
\setAdvisorSgn{}
\setYear{2022}


% С этого момента начинается текст документа
\begin{document}

% Эта команда создает титульную страницу
\makeTitlePage

% Здесь будет автоматически генерироваться содержание документа
\tableofcontents

% Данное окружение оформляет аннотацию: краткое описание текста выделенным абзацем после заголовка
\begin{abstract}
Говорить о крипте можно много, но интереснее знать, как она устроена изнутри и как на ней можно заработать. Эти цели мы поставили в своей работе.


Результатами стали:

\begin{enumerate}

\item Коннекторы для подключения к биржам, чтобы было просто с ними взамодействовать через API.

\item Маркет-мейкинг стратегия.

\item Модель на основе решающих деревьев и градиентного бустинга, предсказывающая направление рынка, базирующаяся на информации об индикаторах.

\item Арбитражная стратегия, которая торгует исходя из курса на другой бирже.

\end{enumerate}


Пайплан нашей работы был такой:

\begin{enumerate}
\item Сначала мы искали торговые стратегии. Рассматривали как самые современные, так и достаточно старые, которые можно переосмыслить с современными технологиями.
\item Потом мы собирали данные для экспериментов. Для этого написали функцию в коннекторе, куда можно подать интересующий интервал времени, и она выдаст датасет с историческими данными.
\item Далее мы очищали данные от мусора, чтобы эксперимент сделать как можно точнее
\item И на самом интересном шаге -- эксперименте -- мы прототипировали стратегию в юпитер-ноутбуке и прогоняли на исторических данных, чтобы понять, можно ли что-то заработать или нет.
\item На последнем шаге мы переписывли нашу стратегию из юпитер-ноутбуков в питоновские файлы, оптимизировали стратегию, чтобы она работало как можно быстрее и занимала как можно меньше памяти. Нам было важно писать код так, чтобы его можно было переиспользовать в других стратегиях, и не делать одну и ту же работу несколько раз.
\end{enumerate}


\end{abstract}

\section{Введение}

\subsection{Актуальность проблемы}
Все говорят о криптовалютах, но толком никто ничего о них не знает. Мы решили разобраться, как устроен блокчейн и технологии вокруг него, какие есть проблемы и как с ними справляются.

Одна из проблем больших блокчейнов, таких как Bitcoin и Ethereum, -- высокая загруженность. Они критически не справляются с нагрузкой. Из-за того, что сеть Эфириума может обрабатывать лишь 15 транзакций в секунду, комиссия, которую нужно заплатить, чтобы транзакция была одной из этих 15, доходит до \$100, что делает любой токен на блокчейне непригодным для использования в качестве обменной валюты.


Чтобы снизить нагрузку на мэйннет\footnote{Основная сеть блокчейна, на которой криптовалюта имеет реальную стоимость. Есть также сети для тестирования разработок. На них валюту можно получить по запросу от специальных адресов}, были разработаны и все еще разрабатываются несколько альтернативных решений, которые одним словом называются \texttt{Layer-2} решения. Это надстройки над блокчейном, которые увеличивают пропускную способность и скорость в ущерб децентрализованности.


Мы считаем, что пока не придумали более изящного способа достичь тех же результатов, которые дают L2 решения, данная технология будет развиваться, а актуальности нашей темы будет расти.


\subsection{Цели и задачи}
\subsubsection{Цель}
Написать трейдинг систему, которая сможет стабильно выходить в плюс.
\subsubsection{Задачи}
\begin{itemize}
\item Проведение исследований по стратегиям трэйдинга.
\item Проверка работоспособность стратегий.
\item Создание инфраструктуры, позволяющей взаимодействовать с биржей автоматизированно.
\item Сбор данных и обучение модели.
\item Написание программы, совершающей сделки.
\item Обзор и сравнительный анализ источников и аналогов
\end{itemize}
К сожалению, выбранная нами тема мало освещается в источниках любого вида: никто не захочет делиться прибыльной стратегией. Многое нам приходилось и придется делать с нуля.
\subsection{Статьи}
Тем не менее, существуют статьи, описывающие некоторые возможные подходы к написанию алгоритмов HFT. Например, есть ресурс~\cite{HftBattle}, на котором описывается стратегия маркет-мейкинга, аналог которой мы попытались реализовать. Но материалы такого рода, находящиеся в открытом доступе, с течением времени теряют свою актуальность: если большое количество участников рынка придерживается одной схемы действий, то вскоре она перестает приносить прибыль. По этой причине мы старались не ориентироваться на подобные источники.
\subsection{Документация бирж}
Основным же источником информации для нас служила документация API~\cite{DydxDocs}~\cite{BinanceDocs} бирж, к которым мы подключались. С помощью нее был написан коннектор, инкапсулирующий процесс подключения и взаимодействия с биржей, произведен сбор необходимой информации: список сделок за последний месяц, состояние о счете и т.п.
\subsection{Документация библиотек}
Для машинного обучения мы использовали CatBoost~\cite{CatboostDocs} — это библиотека от Яндекса для градиентного бустинга, надстройки над решающими деревьями. КэтБуст для нас лучшее решение, потому что это самая быстрая библиотека для классификации среди аналогов и проста в использовании.

\subsection{Аналоги}
На крипто валютном рынке существует множество торговых ботов, но информации об их характеристиках и принципах работы практически нет. Мы можем судить об их доходности, лишь по каким-то сомнительным заявлениям или косвенным признакам. В открытом доступе в основном находятся боты, которые предоставляют лишь интерфейс взаимодействия с биржей ~\cite{CryptoTradingBot}~\cite{FreqTrade}: “ручная” покупка и продажа токенов, выставление лимитных ордеров и т.п.. Такие решения не представляют для нас никакого интереса.

Так достойных аналогов в свободном доступе нет, функциональные и нефункциональные требования нам пришлось придумывать самим. К счастью, они достаточно интуитивны.

\subsection{Описание функциональных требований к программному проекту}

\subsubsection{Коннектор к бирже}

В программе должны быть коннекторы к биржам. Это класс, в конструктор которого подаются приватные ключи кошельков. После этого можно работать с биржей: смотреть информацию о счете, валютаю, отправлять и отменять ордера. Надо реализовать функционал, который предоставляет апи, чтобы можно было его использовать в трейдинг-стратегиях.

Примеры использования функций класса, который способен взаимодействовать с биржами по API:

\begin{itemize}

\item Отправка ордеров
\begin{verbatim}
connector.send_order(
        symbol=ETH_USD, side=BUY, price=1, quantity=0.1
)
\end{verbatim}

\item Получение текущих позиций
\begin{verbatim}
connector.get_our_positions(
        opened=True, symbol=ETH_USD
)
\end{verbatim}

\item Информация о конкретном рынке
\begin{verbatim}
connector.get_symbol_info(market=ETH_USD)
\end{verbatim}

\end{itemize}

\subsubsection{Сбор данных}
Нужно обеспечить удобный механизм сбора исторических данных с бирж, на которых будет тестироваться и обучаться система. Удобно реализовать такие функции в коннекторе, ведь у него уже есть соединение с биржей.

Пример функции для сбора данных:
\begin{itemize}
\item Получение трейдов за определенный промежуток времени
\begin{verbatim}
connector.get_historical_trades(
        market=BTC_USD,
        begin="2021-12-12 09:00:00",
        end="2021-12-12 12:00:00"
)
\end{verbatim}
\end{itemize}

Также в коннекторе может быть функционал, позволяющий собирать данные в прямом эфире и сохранять их в архив, если эти исторические данные не предоставляет биржа.

\subsubsection{Измерение скорости соединения}
Так как счет идет на миллисекунды, мы всегда должны иметь четкое представления, какое время у нас займет отправка и получения пакета данных. Для этого должен быть предусмотрен отдельный модуль, который будет замерять скорость соединения с различными сервисами. Он тоже основывается на коннекторе, так как он уже умеет соединяться с биржей.

Примеры использования функций класса для измерения скорости коннектора:
\begin{itemize}
\item Измерение скорости
\begin{verbatim}
speed_measure = SpeedMeasure(connector)
speed_measure.get_connector_funcs_exec_times(
        market=ETH_USD,
        side=BUY,
        iters_num=10,
        filename="connector_funcs_exec_times.json",
    )
\end{verbatim}

\item Измерение задержки до биржи
\begin{verbatim}
speed_measure.get_orders_processing_delays(
        market=ETH_USD,
        side=BUY,
        orders_num=10,
        filename="orders_processing_delays.json",
)
\end{verbatim}
\end{itemize}

\subsubsection{Машинное обучение}
Индикаторов, по которым можно строить прогнозы, очень много, поэтому ручными методами не получится подобрать правильное соотношение весов. Здесь нужно машинное обучение, которое принимает на вход пред обработанные данные сделок, а на выход выдает модель, которую можно использовать в трейдинг-стратегии. Важно, чтобы модель была устойчива к тому, что в датасет добавляется или убирается индикаторы.
Примеры использования

\subsection{Описание нефункциональных требований к программному проекту}

\subsubsection{Безопасность}
Финансы — чувствительная тема, поэтому наша программа не должна допускать утечек данных о кошельках и приватных ключах. Нужно обеспечить безопасное хранение.

\subsubsection{Отказоустойчивость}
Во время трейдинга торговая система получает сотни обновлений от разных бирж, их обрабатывает, строит прогнозы и торгует. Нужно сделать так, чтобы система была готова к большим нагрузкам, и поведение было однозначно определено. Еще нужно проработать быстрое отключение торговой системы от торгов, если ее поведение станет неадекватным, и можно было бы быстро закрыть открытые заявки.

\subsubsection{Скорость}
В трейдинге важна каждая миллисекунда, поэтому цель — минимизировать время обработки, отправки и принятия данных.

\subsubsection{Переиспользование кода}
Нужно выстроить архитектуру проекта так, чтобы можно было быстро и легко тестировать свои гипотезы, поэтому код, который есть в проекте, должен быть написан так, чтобы его можно было легко понять и переиспользовать в других местах.

\subsubsection{Масштабируемость}
Система должна быть расширяема на несколько бирж и потенциально работать с большим количеством предсказательных моделей.

\section{Market Making}
\subsection{Что это такое}

Когда человек приходит на биржу, он хочет купить актив по рыночной цене.
Если биржа $A$ продает биткоины по 11k\$, а биржа $B$ по 10k\$, то выгоднее покупать биткоины у $B$. Чтобы $A$ не терять клиентов, она пользуется услугами маркет-мейкеров.

Маркет-мейкеры решают проблему дисбаланса цен между биржами. Они могут за вознаграждение от биржи $A$ продать у них биткоины  и выровнять курс до 10k\$, и тогда всем остальным будет снова выгодно торговать на бирже $A$. Аналогично работает, когда на бирже $A$ курс ниже относитьно других бирж: маркет-мейкеры закупятся биткоинами. В стабильное время маркет-мейкеры занимаются поддержанием маленького спреда, то есть делают так, что цены покупки и продажи отличаются как можно меньше.


На децентрализованных биржах все еще интереснее.
Так как там биткоин не привязан к бирже, то стратегия выше выгодня для маркет-мейкеров, потому что они могут на бирже $A$ продать биткоины по 11k\$, и купить на $B$ по 10k\$, заработав разнице в ценах, пока они не выровняются. 

Однако межбиржевые операции очень дорогие и долгие, поэтому чаще всего маркет-мейкеры зарабатывают на резких инертных скачках рынка и вознаграждения от биржи.

Для второго нужно очень много денег, так что попробуем заработать на первом.

\subsection{Как это работает}
Заработать можно в предположении краткосрочно высоко-инерционного рынка.

\definition \textbf{Пробитием} будем называть ту заявку, которую мы не успели отменить, и по ней у нас открылась позиция.

\begin{algorithm}
Market-Making orders
\begin{enumerate}

    \item \label{mm:init} Выставляем лимитные заявки в обе стороны на $\pm \Delta_1$ от индекс-цены.
    
    \item \label{mm:swap} Когда индекс-цена изменяется, переставляем заявки на ту же самую $\pm \Delta_1$, но уже от новой цены.
    
    \item Если переставиться успели, и нас не пробили, то переходим к шагу \ref{mm:swap}
    
    \item Неумаляя общности нас пробили на покупку по цене $p$. Отменяем все заявки
    
    \item Выставляем заявку на продажу на $\Delta_2$ от цены, по которой нас пробили на покупку
    
    \item \label{mm:win} Когда заявка исполнилась, возвращаемся к шагу \ref{mm:init}
    
\end{enumerate}
\end{algorithm}

В итоге на шаге \ref{mm:win} мы заработает $\Delta_2 \cdot p$.

\subsection{Реализация}
\href{https://github.com/dexety/dex-trading-system/tree/main/research/lp-0003-market-making}{Стратегия в репозитории}

Программа работает в трех потоках, которые нужны для: 

\begin{enumerate}

\item Выставление ордеров
\begin{enumerate}
    \item В этом потоке мы получаем новый трейды.
    \item Если цена нового трейда не такая, как у прошлого, и цена ордера, который надо выставить не такая, как у нас сейчас стоит, то мы отменяем текущие ордера и выставляем новые.
    \item Перестановка ордеров переходит не через сначала отмену, старых, а потом выставления новых ордеров, а непосредственно через переставление позиций аргументом \texttt{cancel\_id} в функции \texttt{create\_order}. Это позволило сократить количество запросов в два раза, а значит ошибка \texttt{Too many requests} будет встречаться реже.
\end{enumerate}

\item Получение обновлений ордербука
\begin{enumerate}
    \item В предыдущем потоке во время выставления и отмены ордеров новые обновления по вебсокету не приходят, но обновления ордер бука нам нужны, потому что по нему мы определяем цену оредров.
    \item Поэтому получения ордер бука вынесено в отдельный поток, чтобы всегда именть свежий стакан.
\end{enumerate}

\item Проверки положение ордеров
\begin{enumerate}
    \item Бывает так, что новые трейды не приходят, но окно спреда ордер бука меняется.
    \item Нам важно наше окно трейдов держать строго на $\pm$ какой-то спред вокруг бидсов и асков.
    \item Поэтому мы раз в какое-то время проверяем этот баланс, и если происходит дисбаланс, то мы обновляем наши ордера.
\end{enumerate}

\end{enumerate}

\subsection{Результаты}
Мы торговали на бирже \texttt{DyDx}. Она входит в топ самых популярных децентрализованных бирж. Количество сделок в месяц в ней около 30 тысяч и по биткоинку, и по эфиру. На самом деле это очень мало: в среднем $0.3$ сделки в секунду, то есть одна сделка в 3-4 секунды. Соответственно рынок совсем не высоко-инерционный, поэтому после того, как нас пробили в одну сторону, до пробития во вторую, может пройти несколько десятков минут. Это превышает наш таймаут, потому что если мы слишком долго будем держать открытую позицию, то повышается вероятность, что рынок пойдет в обратную сторону и мы потеряем еще больше денег.

Еще часто бывали случаи, когда рынок отскакивает в нужную нам сторону, но на величину меньше комиссии, соответственно, мы теряем деньги.

\subsection{Выводы}

Если бы у нас была нулевая или близка к нулевой комиссия, то мы бы были в плюсе. Но с той, что у нас есть, в минусе.

\section{Индикаторы}

После маркет-мейкинга мы решили построить модель машинного обучения для предсказания направления рынка. Для построения модели мы использовали индикаторы, которые зависят от трэйдов в окне. Класс \href{https://github.com/dexety/dex-trading-system/blob/ca0370d602f2dfa05262b9b8574002f965ac1502/utils/indicators.py#L5}{\texttt{Indicators}} содержит функции, заполняющие значения фичей и столбца таргета: \texttt{fill\_features\_values} и \texttt{fill\_target\_values}.

\subsection{\texttt{fill\_target\_values}}
\href{https://github.com/dexety/dex-trading-system/blob/ca0370d602f2dfa05262b9b8574002f965ac1502/utils/indicators.py#L15}{Функция в репозитории}

Функция принимает словарь для заполнения значений, два окна трэйдов и параметры ордеров \texttt{stop\_profit} и \texttt{stop\_loss}. Функция считает, исполнились бы эти ордеры и выставляет соответствующее значение в словарь для заполнения.

\subsection{\texttt{fill\_features\_values}}
\href{https://github.com/dexety/dex-trading-system/blob/ca0370d602f2dfa05262b9b8574002f965ac1502/utils/indicators.py#L48}{Функция в репозитории}

\begin{designation}
Окно длины $t$ -- окно, в котором есть все трэйды, случившиеся не больше чем за $t$ секунд до какого-то момента. Обозначаем как \textit{window}.
\end{designation}

\begin{designation}
    \texttt{window[i]} -- i-ый элементы в окне.
\end{designation}
\begin{designation}
    \texttt{window[i].price} и \texttt{window[i].volume} -- цена и объем i-й сделки в окне соответсвенно.
\end{designation}

Функция принимает словарь для заполнения, окно трэйдов, и два списка вариантов числовых параметров фичей $n$ и $t$. За $n$ всегда будем обозначать количество сделок в окне. Функция заполняет словарь значениями фичей, которые представлены ниже:

\begin{enumerate}
    \item \texttt{seconds\_since\_midnight} -- количество секунд с начала дня.
    \item \texttt{seconds\_since\_n\_trades\_ago} -- количество секунд, прошедших с первого трэйда в окне из $n$ последних трэйдов.
    
    \textbf{Псевдо-формула:} \texttt{(window.end\_time - window[-n].time).to\_seconds()}

    \item \texttt{WI\_exp\_moving\_average} -- экспоненциальное среднее в окне длины $t$. 
    
    \textbf{Псевдо-формула:} \texttt{$\alpha$ (window[-1].price + $(1 - \alpha)$ window[-2].price + \dots + $(1 - \alpha)^{n - 1}$ window[-n].price)}

    \item \texttt{WI\_weighted\_moving\_average} -- взвешанное среднее в окне длины $t$.

    \textbf{Псевдо-формула:} $\frac{\sum\limits_{i=0}^{n} \texttt{window[i].price} \; \cdot \; \texttt{window[i].volume}}{\sum\limits_{i=0}^{n}\texttt{window[i].size}}$

    \item \texttt{WI\_trade\_amount} -- количество сделок в окне длины $t$.

    \textbf{Псевдо-формула:} n

    \item \texttt{WI\_trade\_volume} -- суммарный объем сделок в окне длины $t$.

    \textbf{Псевдо-формула:} $\sum\limits_{i=0}^{n}\texttt{window[i].volume}$

    \item \texttt{WI\_open\_close\_diff} -- \textbf{частное} между ценой закрытия и ценой открытия в окне длины $t$.

    \textbf{Псевдо-формула:} \texttt{window[-1].price / window[0].price}

    \item \texttt{WI\_stochastic\_oscillator} -- стохастический осцилятор.

    \textbf{Псевдо-формула:} $\frac{\texttt{CUR} - \texttt{MIN}}{\texttt{MAX} - \texttt{MIN}}$, где \texttt{CUR} - цена последней сделки, а \texttt{MIN} и \texttt{MAX} - наименьшая и наибольшая цена сделки соответственно в окне длины $t$.

\end{enumerate}

Вторая фича зависит от значения $n$ и от направления трэйдов в окне, поэтому для каждой комбинации направления и параметра $n$ в таблице будет свой столбец.

Последние 5 фичей зависят от $t$ и от направления трэйдов в окне, поэтому для каждой комбинации направления, параметра $t$ и фичи в таблице будет свой столбец.

В послдедствии мы решили еще дополнительно делить все скользящие средние на среднее арифметическое цен всех сделок в окне. 

\textbf{Псевдо-формула:} $\texttt{WI\_moving\_average} \; \cdot \; \frac{n}{\sum\limits_{i=0}^{n - 1} window[i].price}$ 

Это сделано для того, чтобы при глобальном изменении стоимости валюты модель продолжала работать.

\section{Трейдер}

\href{https://github.com/dexety/dex-trading-system/blob/main/strategy/arbitrage/trader.py}{Трейдер в репозитории}

\begin{definition}
\textbf{Трейдер} -- программа, которая на основе готовой стратегии, совершает сделки.
\end{definition}

\subsection{Использованные технологии}

\subsubsection{\texttt{Асинхронная функция}}

\begin{definition}
\textbf{Асинхронная функция} -- функция, которая не блокирует основой поток программы.
\end{definition}

Наш проект написан на питоне, а он одно-поточный. Поэтому нужно было пользоваться всеми возможностями асинхронности. Для этого в питоне есть очень простой интерфейс:

\begin{verbatim}
async def slow_calculation():
    return ...
    
async def hard_calculation():
    return ...
    
def main():
    loop = asyncio.get_event_loop()
    loop.create_task(
        slow_calculation(), name="slow_calculation"
    )
    loop.create_task(
        hard_calculation(), name="hard_calculation"
    )
    loop.run_forever()
\end{verbatim}

Здесь мы создали две задачи: \texttt{slow\_calculation} и  \texttt{hard\_calculation}, которые будут выполняться асинхронно, то есть не мешать друг другу своим вычислениями.

\subsubsection{\texttt{asyncio.Event}}

В трейдере, помимо асинхронных функций, используется класс \texttt{asyncio.Event}, позволяющий ждать задачам, пока событие не произойдет в другой задаче и элемент класса не будет разблокирован.

\begin{verbatim}
async def waiter(event):
    print('waiting for it ...')
    await event.wait()
    print('... got it!')

async def main():
    # Create an Event object.
    event = asyncio.Event()

    # Spawn a Task to wait until 'event' is set.
    waiter_task = asyncio.create_task(waiter(event))

    # Sleep for 1 second and set the event.
    await asyncio.sleep(1)
    event.set()

    # Wait until the waiter task is finished.
    await waiter_task

asyncio.run(main())
\end{verbatim}

В данном примере функция \texttt{waiter} не завершится до тех пор, пока функция \texttt{main} не разблокирует событие строчкой \texttt{event.set()}, потому что \texttt{waiter} находится в ожидании после строчки \texttt{await event.wait()}.

С помощью этого класса можно синхронизировать работу алгоритма торговли с ответами от биржи.

\subsection{Устройство трейдера}

Используемые параметры:
\begin{itemize}
    \item \href{https://github.com/dexety/dex-trading-system/blob/ca0370d602f2dfa05262b9b8574002f965ac1502/research/ib-0002-cross-analysis/trader.py#L52}{\texttt{trailing\_percent}} -- процент, на который цена stop-loss ордера отстает от рыночной при отправке trailing-stop ордера.
    \item \href{https://github.com/dexety/dex-trading-system/blob/ca0370d602f2dfa05262b9b8574002f965ac1502/research/ib-0002-cross-analysis/trader.py#L54}{\texttt{profit\_threshold}} -- процент, на который цена лимитки, выставленной по take-profit ордеру отличается от рыночной.
    \item \href{https://github.com/dexety/dex-trading-system/blob/ca0370d602f2dfa05262b9b8574002f965ac1502/research/ib-0002-cross-analysis/trader.py#L55}{\texttt{sec\_to\_wait}} -- количество секунд, через которое цикл торговли считается завершенным, если ни take-profit, ни trailing-stop ордеры не были исполнены.
\end{itemize}

Трейдер состоит из двух асинхронных задач и вспомогательных функций. 

Из ключевых вспомогательных функций нужно отметить функцию \href{https://github.com/dexety/dex-trading-system/blob/main/research/ib-0002-cross-analysis/trader.py#L303}{\texttt{reset}}, которая переводит все поля класса в состояние для ожидания скачка на бинансе. Она вызывается после успешного завершения цикла торговли, или после отмены ордеров и закрытия позиции в случае, когда какой либо ордер не смог исполниться.

Основные асинхронные задачи:
\begin{itemize}
    \item \texttt{listen\_binance}
    \item \texttt{account\_listener}
\end{itemize}

Пройдемся по каждой

\subsubsection{\texttt{listen\_binance}}

\href{https://github.com/dexety/dex-trading-system/blob/main/research/ib-0002-cross-analysis/trader.py#L167}{Функция в репозитории}

\begin{definition}
\textbf{Слушать [что-то]} -- значит подключиться по веб-сокету к какому-то серверу и получать от него обновления.
\end{definition} 

Эта функция реализует логику торговли. В этом трейдере по техническим причинам не получилось использовать функционал коннектора для бинанса, поэтому подписываться на обновления пришлось в ручную. Функция держит окно трейдов на одном направлении фиксированной длины, и если перепад цены превысил порог, то запускается алгоритм торговли:

\begin{enumerate}
    \item Отправляется маркет ордер. Функция входит в режим ожидания с помощью \texttt{asyncio.Event} и ждет, пока с биржи придет ответ об исполнении ордера. Если с биржи приходит сообщение об отмене ордера, то вызывается функция \texttt{reset} и трэйдер возвращается в состояние ожидания скачка цены на бинансе.
    \item Когда \texttt{account\_listener} сообщил об успешном выполнении маркет ордера и сохранил цену по которой он исполнился, трейдер выставляет \texttt{take-profit} и \texttt{trailing-stop} ордеры с заданными при инициализации класса параметрами и на протяжении \texttt{sec\_to\_wait} секунд ожидает исполнения хотя бы одного из них.
    \begin{itemize}
        \item Если за \texttt{sec\_to\_wait} секунд ордеры не закрылись, то они отменяются. Далее закрывается позиция в которую мы зашли маркетом, и когда с биржи приходит сообщение о закрытии позиции, цикл торговли считается завершенным.
        \item Если какой-то из ордеров исполнился, то мы в любом случае отменяем оставшийся. Если исполнился лимит, то мы в плюсе, если трэйлинг стоп, то, скорее всего, в минусе. После отмены цикл считается завершенным.
    \end{itemize}
    
\end{enumerate}


\subsubsection{\texttt{account\_listener}}

\href{https://github.com/dexety/dex-trading-system/blob/main/research/ib-0002-cross-analysis/trader.py#L365}{Функция в репозитории}

Эта задача слушает изменения в наших ордерах и позициях, обновляет цену, по которой исполнился маркет, выставляет флаги и разблокирует нужные события, на которых спит функция \texttt{listen\_binance}, тем самым обеспечивая её синзронизацию с биржей.


\section{Описание функциональных требований к программному проекту}

\subsection{Сбор данных}
Нужно обеспечить удобный механизм сбора исторических данных с бирж, на которых будет тестироваться и обучаться система.

\subsection{Измерение скорости соединения}
Так как счет идет на миллисекунды, мы всегда должны иметь четкое представления, какое время у нас займет отправка и получения пакета данных. Для этого должен быть предусмотрен отдельный модуль, который будет замерять скорость соединения с различными сервисами.

\subsection{Коннектор к бирже}
В программе должны быть коннекторы к биржам. Это класс, в конструктор которого подаются приватные ключи кошельков. После этого можно работать с биржей: смотреть информацию о счете, валютаю, отправлять и отменять ордера. Надо реализовать функционал, который предоставляет апи, чтобы можно было его использовать в трейдинг-стратегиях.

\subsection{Машинное обучение}
Индикаторов, по которым можно строить прогнозы, очень много, поэтому ручными методами не получится подобрать правильное соотношение весов. Здесь нужно машинное обучение, которое принимает на вход пред обработанные данные сделок, а на выход выдает модель, которую можно использовать в трейдинг-стратегии. Важно, чтобы модель была устойчива к тому, что в датасет добавляется или убирается индикаторы.
Примеры использования

\subsection{Коннектор}
Предоставляем класс, который способен взаимодействовать с биржами по API, например:

\begin{itemize}

\item Отправка ордеров
\begin{lstlisting}[language=Python]
connector.send_order(
        symbol=ETH_USD, side=BUY, price=1, quantity=0.1
)
\end{lstlisting}

\item Получение текущих позиций
\begin{lstlisting}[language=Python]
connector.get_our_positions(
        opened=True, symbol=ETH_USD
)
\end{lstlisting}

\item Получение трейдов за определенный промежуток времени
\begin{lstlisting}[language=Python]
connector.get_historical_trades(
        market=BTC_USD,
        begin="2021-12-12 09:00:00",
        end="2021-12-12 12:00:00"
)
\end{lstlisting}

\item Информация о конкретном рынке
\begin{lstlisting}[language=Python]
connector.get_symbol_info(market=ETH_USD)
\end{lstlisting}

\item Измерение скорости
\begin{lstlisting}[language=Python]
speed_measure = SpeedMeasure(connector)
speed_measure.get_connector_funcs_exec_times(
        market=ETH_USD,
        side=BUY,
        iters_num=10,
        filename="connector_funcs_exec_times.json",
    )
\end{lstlisting}

\item Измерение задержки до биржи
\begin{lstlisting}[language=Python]
speed_measure.get_orders_processing_delays(
        market=ETH_USD,
        side=BUY,
        orders_num=10,
        filename="orders_processing_delays.json",
)
\end{lstlisting}

\end{itemize}

\section{Описание нефункциональных требований к программному проекту}

\subsection{Безопасность}
Финансы — чувствительная тема, поэтому наша программа не должна допускать утечек данных о кошельках и приватных ключах. Нужно обеспечить безопасное хранение.

\subsection{Отказоустойчивость}
Во время трейдинга торговая система получает сотни обновлений от разных бирж, их обрабатывает, строит прогнозы и торгует. Нужно сделать так, чтобы система была готова к большим нагрузкам, и поведение было однозначно определено. Еще нужно проработать быстрое отключение торговой системы от торгов, если ее поведение станет неадекватным, и можно было бы быстро закрыть открытые заявки.

\subsection{Скорость}
В трейдинге важна каждая миллисекунда, поэтому цель — минимизировать время обработки, отправки и принятия данных.

\subsection{Переиспользование кода}
Нужно выстроить архитектуру проекта так, чтобы можно было быстро и легко тестировать свои гипотезы, поэтому код, который есть в проекте, должен быть написан так, чтобы его можно было легко понять и переиспользовать в других местах.

\subsection{Масштабируемость}
Система должна быть расширяема на несколько бирж и потенциально работать с большим количеством предсказательных моделей.

\section{Дополнительный результаты}

Помимо задач, которые нам поставил руководитель, мы сделали еще:

\subsection{Визуализация}
Перед тем, как работать с данными, надо понять, как они устроены: попарное распределение классов, плотность каждого из индикаторов. Мы все это сделали. Теперь стало гораздо проще подбирать параметры для модели машинного обучения.

\subsection{Маркет-Мэйкинг стратегия}
Это стратегия, когда мы держим окно из зеркальных заявок на каком-то уровне от индекс-цены. Утверждается, что если нас пробили с одной стороны, то почти сразу пробьют и со второй стороны, и мы окажемся в плюсе. Но все оказалось не так: при пробитии нас с одной стороны, рынок продолжал идти в ту же сторону.

\subsection{Контролирующий бот}
Мы сделали телеграм бота, на которого можно по паролю подписаться и получать обновления состояния аккаунта на бирже. Это полезно, когда ты запускаешь торговую стратегию, куда-то отходишь, но при этом всегда можешь контролировать, что с ней происходит, через телеграм.

\subsection{CI/CD, тесты, линтер}
Любая ошибка в торговой системе может потерять наши деньги, поэтому важно, чтобы весь код всегда был рабочим. Для этого мы все, что смогли, обложили тестами c использованием утилиты PyTest~\cite{Pytest}. Теперь при каждом пулл реквесте у нас запускается тестирующая система, и если какие-то тесты не проходят, то мы запрещаем дальнейший пуш. Реализовали мы это через GitHub Actions~\cite{GitHubActions}.

Еще нам хочется, чтобы весь код был консистеным. Для этого мы используем линтер Black~\cite{Black} и статический анализатор PyLint~\cite{Pylint}. В совокупности эти утилиты поддерживают консистентность нашего кода и могут еще до запуска теста, выдать синтаксические ошибки.

\subsection{Смарт контракты}
С помощью смарт контрактов можно занимать у других людей миллиарды, трейдить на них, и потом возвращать криптовалюту обратно. Звучит заманчиво. Мы написали смарт конракты на Solidity~\cite{Solidity}, и залили их в сеть эфира через Brownie~\cite{Brownie}.


% Здесь автоматически генерируется библиография. Первая команда задает стиль оформления библиографии, а вторая указывает на имя файла с расширением bib, в котором находится информация об источниках.
\bibliographystyle{plainurl}
\bibliography{bibl}

% Здесь текст документа заканчивается
\end{document}
% Начиная с этого момента весь текст LaTeX игнорирует, можете вставлять любую абракадабру.
